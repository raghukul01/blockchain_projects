
%=======================   Default Templete   ==================
\documentclass[a4paper]{article}


% file with some default definations
\input{structure.tex}
\usepackage{listings}
\lstset{language=Python, basicstyle=\normalsize\sffamily\linespread{0.8}, numbers=left, numberstyle=\small, stepnumber=1, numbersep=5pt}
\usepackage{fancyhdr}
\usepackage{tikz}
\usepackage{tikz-qtree}
\usepackage{mathtools}
\DeclarePairedDelimiter{\ceil}{\lceil}{\rceil}

\setlength{\parindent}{0pt}

\pagestyle{fancy}
\fancyhf{}
\lhead{\textbf{\NAME\ (\ANDREWID)}}
\chead{\textbf{Mid Semester Exam \HWNUM}}
\rhead{\COURSE}

\renewcommand{\qedsymbol}{\rule{0.7em}{0.7em}}

%==================Header details======================
\newcommand\NAME{Raghukul Raman}
\newcommand\ANDREWID{160538}
\newcommand\HWNUM{}
\newcommand\COURSE{CS731}
%======================================================

% available formatted sections:
% - COMMAND LINE ENVIRONMENT: \begin{commandline} \end{commandline}
% - FILE CONTENTS ENVIRONMENT: \begin{file}[optional filename, defaults to "File"]
% - NUMBERED QUESTIONS ENVIRONMENT: \begin{question}[optional title]
% - WARNING TEXT ENVIRONMENT(can also be used for note): \begin{warn}[optional title, defaults to "Warning:"]
% - INFORMATION ENVIRONMENT(can be used to mention given details): \begin{info}[optional title, defaults to "Info:"]

%===============================================================
\begin{document}

\subsection*{Problem 1:}
Two suggested protocols to increase transaction rate to $70$/second are:
\begin{enumerate}
    \item Increase size of block from $1$MB to $10$MB.
    \item Decrease the block interval time to $1$ min.
\end{enumerate}
\textbf{Issues with increasing block size:} This would require a lot more storage, and would make the full
            nodes more expensive to operate. And eventaully fully operating nodes will decrease, and the system
            will become more centralized. There might be more double spending attacks due to slower propogation
            speeds. Achieving consensus would be difficult since validating a block would require lot more efforts. \\

\textbf{Issues with decreasing block interval time:} Propogation of the block in the network, takes some time. 
            If we reduce the block interval time, block might not be propogated fully in the network till that time.
            We might not be able to achieve consensus in due time. Block propogation and latency would also lead to more
            orphan nodes, since they might be delayed in propogating. It would also have an envionmental effect, since
            power would be used more since block rate is high.

\begin{question}
    % Q2
\end{question}



\begin{question}
    % Q3
\end{question}
He is wrong. We can have a double spending attact is there are $51$

\begin{question}
    % Q4
\end{question}



\begin{question}
    % Q5
\end{question}

\begin{question}
    % Q6
\end{question}



\begin{question}
    % Q7
\end{question}

\begin{question}
    % Q8
\end{question}

\begin{question}
    % Q9
\end{question}

\end{document}
