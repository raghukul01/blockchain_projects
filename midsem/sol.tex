
%=======================   Default Templete   ==================
\documentclass[a4paper]{article}


% file with some default definations
%%%%%%%%%%%%%%%%%%%%%%%%%%%%%%%%%%%%%%%%%
% Lachaise Assignment
% Structure Specification File
% Version 1.0 (26/6/2018)
%
% This template originates from:
% http://www.LaTeXTemplates.com
%
% Authors:
% Marion Lachaise & François Févotte
% Vel (vel@LaTeXTemplates.com)
%
% License:
% CC BY-NC-SA 3.0 (http://creativecommons.org/licenses/by-nc-sa/3.0/)
% 
%%%%%%%%%%%%%%%%%%%%%%%%%%%%%%%%%%%%%%%%%

%----------------------------------------------------------------------------------------
%	PACKAGES AND OTHER DOCUMENT CONFIGURATIONS
%----------------------------------------------------------------------------------------

\usepackage{amsmath,amsfonts,stmaryrd,amssymb} % Math packages

\usepackage{amsthm}

\usepackage{enumerate} % Custom item numbers for enumerations

\usepackage[ruled]{algorithm2e} % Algorithms

\usepackage[framemethod=tikz]{mdframed} % Allows defining custom boxed/framed environments

\usepackage{listings} % File listings, with syntax highlighting
\lstset{
	basicstyle=\ttfamily, % Typeset listings in monospace font
}

%----------------------------------------------------------------------------------------
%	DOCUMENT MARGINS
%----------------------------------------------------------------------------------------

\usepackage{geometry} % Required for adjusting page dimensions and margins

\geometry{
	paper=a4paper, % Paper size, change to letterpaper for US letter size
	top=2.5cm, % Top margin
	bottom=3cm, % Bottom margin
	left=2.5cm, % Left margin
	right=2.5cm, % Right margin
	headheight=14pt, % Header height
	footskip=1.5cm, % Space from the bottom margin to the baseline of the footer
	headsep=1.2cm, % Space from the top margin to the baseline of the header
	%showframe, % Uncomment to show how the type block is set on the page
}

%----------------------------------------------------------------------------------------
%	FONTS
%----------------------------------------------------------------------------------------

\usepackage[utf8]{inputenc} % Required for inputting international characters
\usepackage[T1]{fontenc} % Output font encoding for international characters

\usepackage{XCharter} % Use the XCharter fonts

%----------------------------------------------------------------------------------------
%	COMMAND LINE ENVIRONMENT
%----------------------------------------------------------------------------------------

% Usage:
% \begin{commandline}
%	\begin{verbatim}
%		$ ls
%		
%		Applications	Desktop	...
%	\end{verbatim}
% \end{commandline}

\mdfdefinestyle{commandline}{
	leftmargin=10pt,
	rightmargin=10pt,
	innerleftmargin=15pt,
	middlelinecolor=black!50!white,
	middlelinewidth=2pt,
	frametitlerule=false,
	backgroundcolor=black!5!white,
	frametitle={Command Line},
	frametitlefont={\normalfont\sffamily\color{white}\hspace{-1em}},
	frametitlebackgroundcolor=black!50!white,
	nobreak,
}

% Define a custom environment for command-line snapshots
\newenvironment{commandline}{
	\medskip
	\begin{mdframed}[style=commandline]
}{
	\end{mdframed}
	\medskip
}

%----------------------------------------------------------------------------------------
%	FILE CONTENTS ENVIRONMENT
%----------------------------------------------------------------------------------------

% Usage:
% \begin{file}[optional filename, defaults to "File"]
%	File contents, for example, with a listings environment
% \end{file}

\mdfdefinestyle{file}{
	innertopmargin=1.6\baselineskip,
	innerbottommargin=0.8\baselineskip,
	topline=false, bottomline=false,
	leftline=false, rightline=false,
	leftmargin=2cm,
	rightmargin=2cm,
	singleextra={%
		\draw[fill=black!10!white](P)++(0,-1.2em)rectangle(P-|O);
		\node[anchor=north west]
		at(P-|O){\ttfamily\mdfilename};
		%
		\def\l{3em}
		\draw(O-|P)++(-\l,0)--++(\l,\l)--(P)--(P-|O)--(O)--cycle;
		\draw(O-|P)++(-\l,0)--++(0,\l)--++(\l,0);
	},
	nobreak,
}

% Define a custom environment for file contents
\newenvironment{file}[1][File]{ % Set the default filename to "File"
	\medskip
	\newcommand{\mdfilename}{#1}
	\begin{mdframed}[style=file]
}{
	\end{mdframed}
	\medskip
}

%----------------------------------------------------------------------------------------
%	NUMBERED QUESTIONS ENVIRONMENT
%----------------------------------------------------------------------------------------

% Usage:
% \begin{question}[optional title]
%	Question contents
% \end{question}

\mdfdefinestyle{question}{
	innertopmargin=1.2\baselineskip,
	innerbottommargin=0.8\baselineskip,
	roundcorner=5pt,
	nobreak,
	singleextra={%
		\draw(P-|O)node[xshift=1em,anchor=west,fill=white,draw,rounded corners=5pt]{%
		Question \theQuestion\questionTitle};
	},
}

\newcounter{Question} % Stores the current question number that gets iterated with each new question

% Define a custom environment for numbered questions
\newenvironment{question}[1][\unskip]{
	\bigskip
	\stepcounter{Question}
	\newcommand{\questionTitle}{~#1}
	\begin{mdframed}[style=question]
}{
	\end{mdframed}
	\medskip
}

%----------------------------------------------------------------------------------------
%	WARNING TEXT ENVIRONMENT
%----------------------------------------------------------------------------------------

% Usage:
% \begin{warn}[optional title, defaults to "Warning:"]
%	Contents
% \end{warn}

\mdfdefinestyle{warning}{
	topline=false, bottomline=false,
	leftline=false, rightline=false,
	nobreak,
	singleextra={%
		\draw(P-|O)++(-0.5em,0)node(tmp1){};
		\draw(P-|O)++(0.5em,0)node(tmp2){};
		\fill[black,rotate around={45:(P-|O)}](tmp1)rectangle(tmp2);
		\node at(P-|O){\color{white}\scriptsize\bf !};
		\draw[very thick](P-|O)++(0,-1em)--(O);%--(O-|P);
	}
}

% Define a custom environment for warning text
\newenvironment{warn}[1][Warning:]{ % Set the default warning to "Warning:"
	\medskip
	\begin{mdframed}[style=warning]
		\noindent{\textbf{#1}}
}{
	\end{mdframed}
}

%----------------------------------------------------------------------------------------
%	INFORMATION ENVIRONMENT
%----------------------------------------------------------------------------------------

% Usage:
% \begin{info}[optional title, defaults to "Info:"]
% 	contents
% 	\end{info}

\mdfdefinestyle{info}{%
	topline=false, bottomline=false,
	leftline=false, rightline=false,
	nobreak,
	singleextra={%
		\fill[black](P-|O)circle[radius=0.4em];
		\node at(P-|O){\color{white}\scriptsize\bf i};
		\draw[very thick](P-|O)++(0,-0.8em)--(O);%--(O-|P);
	}
}

% Define a custom environment for information
\newenvironment{info}[1][Info:]{ % Set the default title to "Info:"
	\medskip
	\begin{mdframed}[style=info]
		\noindent{\textbf{#1}}
}{
	\end{mdframed}
}
\usepackage{listings}
\lstset{language=Python, basicstyle=\normalsize\sffamily\linespread{0.8}, numbers=left, numberstyle=\small, stepnumber=1, numbersep=5pt}
\usepackage{fancyhdr}
\usepackage{tikz}
\usepackage{tikz-qtree}
\usepackage{mathtools}
\usepackage{xcolor}
\definecolor{commentgreen}{RGB}{2,112,10}
\definecolor{eminence}{RGB}{108,48,130}
\definecolor{weborange}{RGB}{255,165,0}
\definecolor{frenchplum}{RGB}{129,20,83}

\usepackage{listings}
\usepackage{xcolor}
\lstdefinestyle{DOS}
{
    backgroundcolor=\color{white},
    basicstyle=\scriptsize\color{black}\ttfamily
}
\lstdefinestyle{C++} {
    language=C++,
    basicstyle=\ttfamily,
    keywordstyle=\color{blue}\ttfamily,
    stringstyle=\color{red}\ttfamily,
    commentstyle=\color{green}\ttfamily,
    morecomment=[l][\color{magenta}]{\#},
}
\DeclarePairedDelimiter{\ceil}{\lceil}{\rceil}

\setlength{\parindent}{0pt}

\pagestyle{fancy}
\fancyhf{}
\lhead{\textbf{\NAME\ (\ANDREWID)}}
\chead{\textbf{Mid Semester Exam \HWNUM}}
\rhead{\COURSE}

\renewcommand{\qedsymbol}{\rule{0.7em}{0.7em}}

%==================Header details======================
\newcommand\NAME{Raghukul Raman}
\newcommand\ANDREWID{160538}
\newcommand\HWNUM{}
\newcommand\COURSE{CS731}
%======================================================

% available formatted sections:
% - COMMAND LINE ENVIRONMENT: \begin{commandline} \end{commandline}
% - FILE CONTENTS ENVIRONMENT: \begin{file}[optional filename, defaults to "File"]
% - NUMBERED QUESTIONS ENVIRONMENT: \begin{question}[optional title]
% - WARNING TEXT ENVIRONMENT(can also be used for note): \begin{warn}[optional title, defaults to "Warning:"]
% - INFORMATION ENVIRONMENT(can be used to mention given details): \begin{info}[optional title, defaults to "Info:"]

%===============================================================
\begin{document}

\subsection*{Problem 1:}
Two suggested protocols to increase transaction rate to $70$/second are:
\begin{enumerate}
    \item Increase size of block from $1$MB to $10$MB.
    \item Decrease the block interval time to $1$ min (from $10$ minute).
\end{enumerate}
\textbf{Issues with increasing block size:} This would require a lot more storage, and would make the full
            nodes more expensive to operate. And eventaully fully operating nodes will decrease, and the system
            will become more centralized. There might be more double spending attacks due to slower propogation
            speeds. Achieving consensus would be difficult since validating a block would require lot more efforts. \\

\textbf{Issues with decreasing block interval time:} Propogation of the block in the network, takes some time. 
            If we reduce the block interval time, block might not be propogated fully in the network till that time.
            We might not be able to achieve consensus in due time. Block propogation and latency would also lead to more
            orphan nodes, since they might be delayed in propogating. It would also have an envionmental effect, since
            power would be used more since block rate is high.

\subsection*{Problem 2:}
\begin{lstlisting}[style=C++]
#include <openssl/conf.h>
#include <openssl/evp.h>
#include <openssl/err.h>
#include <openssl/pem.h>
#include <openssl/crypto.h>
#include <openssl/sha.h>
#include <openssl/rsa.h>

#include <bits/stdc++.h>

// bits denote the modulus size
const int bits = 2048;
// string needed for conversion to base58
static const char* pszBase58 = "123456789ABCDEFGHJKLMNPQRSTUVWXYZ"
                                        +"abcdefghijkmnopqrstuvwxyz";
// Our RSA object to store key
RSA *rsa = NULL;
BIGNUM *bne = NULL;
// Key variables, that would store data in raw format
EVP_PKEY *privateKey, *publicKey;
// public key in DER format
uint8_t* pubKeyDER;
// length of DER public key
int key_len;

void set_up() {
    // standard procedure taken from assignment1 crypto.cpp
    bne = BN_new();
    int ret = BN_set_word(bne, RSA_F4);
    assert(ret==1);

    // generate a new object from the constructor functions
    rsa = RSA_new();
    privateKey = EVP_PKEY_new();
    publicKey = EVP_PKEY_new();
}

void gen_keys() {
    // generates a key pair and stores it in the RSA structure provided in rsa.
    int ret=RSA_generate_key_ex(rsa,bits,bne,NULL);
    assert(ret==1);
    
    // encode and decode a public key.
    key_len = i2d_RSA_PUBKEY(rsa, &pubKeyDER);
    // store rsa value generated in keys.
    EVP_PKEY_assign_RSA(privateKey, rsa);
    EVP_PKEY_assign_RSA(publicKey, rsa);
}

void print_keys() {
    // print keys to standard buffer.
    PEM_write_PrivateKey(stdout, privateKey, NULL, NULL, 0, NULL, NULL);
    PEM_write_PUBKEY(stdout, publicKey);
}

void free_structures() {
    BN_free(bne); // free the big number structure
    free(pubKeyDER); // free DER formatted key
    RSA_free(rsa);
}

/*
CITE:
    Following implementation is taken from: bitcoin github repo
    link: https://github.com/bitcoin/bitcoin/blob/master/src/base58.cpp
*/
std::string EncodeBase58(const unsigned char* pbegin, const unsigned char* pend)
{
    // Skip & count leading zeroes.
    int zeroes = 0;
    int length = 0;
    while (pbegin != pend && *pbegin == 0) {
        pbegin++;
        zeroes++;
    }
    // Allocate enough space in big-endian base58 representation.
    int size = (pend - pbegin) * 138 / 100 + 1; // log(256) / log(58), rounded up.
    std::vector<unsigned char> b58(size);
    // Process the bytes.
    while (pbegin != pend) {
        int carry = *pbegin;
        int i = 0;
        // Apply "b58 = b58 * 256 + ch".
        for (std::vector<unsigned char>::reverse_iterator it = b58.rbegin();
                (carry != 0 || i < length) && (it != b58.rend()); it++, i++) {
            carry += 256 * (*it);
            *it = carry % 58;
            carry /= 58;
        }

        assert(carry == 0);
        length = i;
        pbegin++;
    }
    // Skip leading zeroes in base58 result.
    std::vector<unsigned char>::iterator it = b58.begin() + (size - length);
    while (it != b58.end() && *it == 0)
        it++;
    // Translate the result into a string.
    std::string str;
    str.reserve(zeroes + (b58.end() - it));
    str.assign(zeroes, '1');
    while (it != b58.end())
        str += pszBase58[*(it++)];
    return str;
}
// cite end..

int main() {
    set_up();
    gen_keys();
    print_keys();

    // compute hash of PEM formatted public key.
    // This procedure is taken from Lec 3 slides.
    unsigned char h1[SHA256_DIGEST_LENGTH];
    SHA256_CTX sha256;
    SHA256_Init(&sha256);
    SHA256_Update(&sha256, &pubKeyDER[0], key_len);
    SHA256_Final(h1, &sha256);

    // Encode it in base58 using the standard implementation.
    auto bitcoin_addr = EncodeBase58(h1, h1+SHA256_DIGEST_LENGTH);
    // print the bitcoint address generated.
    std::cout << bitcoin_addr << "\n";

    free_structures();
}
\end{lstlisting}
Public key generated in this way is:
\begin{lstlisting}
-----BEGIN PUBLIC KEY-----
MIIBIjANBgkqhkiG9w0BAQEFAAOCAQ8AMIIBCgKCAQEAuqdKYAffbvISUIzFlMQY
FrJP/oyaTF8K+t8dynVKXR+0wgby8D68l6+G2fa97lp3gmZ3mfT8NjNskXP5CTL9
tP/x3XkGxYz5fVoByZNa160Udscghe2A2qDtTxJMdRVTDKU5M7UduQxK5iUkjVpi
Ejphrksw7qBLGrP4KCDbunnkZdAE/ns3LESi5SzwZ/aWuutZLJxaAiZaNTOHc9NQ
hXeOkyKC8H2+XlxJFUZsfExNTmhpls4+EE2KYD7Hqk9J4VYE1N6vMCvCC45H7A/B
X4vRtx5kGAv8anJdXH+rRChyQr5RakIQz4fDOhEw8hcIgTx6QXbEOtrxtTlWi3H/
1QIDAQAB
-----END PUBLIC KEY-----
\end{lstlisting}
The bitcoin address generated for this key is:
\begin{lstlisting}
    addr := Cgwj8VL9QLhGg9RG2oHoiVsSPqD8VdUze2coUDgxQWf3
\end{lstlisting}

\subsection*{Problem 3:}
In bitcoin blockchain we can have double spending attacks. Let's see how: \\

Suppose Alice bought cocaine from Bob by paying him in bitcoin. Seeing this transaction in the most
recent block Bob might think that transaction is successful, and would give alway cocaine to Alice.
Now if the next random node this is selected in the next round happens to be controlled by Alice.
So she might ignore the block including her cocaine transaction, and could build on the second most recent block,
not including the cocaine transaction. So next time any miner would see 2 branches of equal length,
being honest it would the would be equally probable to build new block on either of them. Since There
is no way to distinguish that one of the block tries to double spend, there are approximately $50\%$ chances
that double spend would occur.\\

\textbf{$\mathbf{51 \%}$ attack}: If there is a node that has > 51\% hash power, then it can double by
simply growing over the other transaction block, to a large extent, such that even 6 cofirmations won't help.
So in above example if Bob was satisfied after seeing 6 more blocks on top of current transaction's block, Alice
might start a new chain, and since she has hash power > 51\%, she can easily take over the other chain.
% Doubt: should we explain 6 transaction confirmation or this would suffice?

\subsection*{Problem 4:}
\subsubsection*{(a)}
Since the miner is ahead of public blockchain by two secret blocks, all the mining efforts of the rest of the network
will be wasted. Other miners would mine on top of what they think is the longest chain, so after the
selfish miner announces, that branch would instantly become the new longest chain. Eventaully 
the rest of block (found by other miners), would become orphan. So in nutshell, gain in selfish mining
is that effective share of mining rewards would increase.

\subsection*{(b)}
First of all using time stamp is not a good way since we would encounter network latency, and it would
not be synchronous over the blockchain network. But ignoring them, we can get an idea of selfish miners, but
we can never be $100\%$ sure that the current miner, who's timestamp are close is a selfish miner. Since there
might be a miner who was lucky enough to find the next block early. So we cannot define a hard and fast rule to detect
selfish miner.




\subsection*{Problem 5:}
\subsection*{(a)}
For each miner we know that probability of successfully solving puzzle = $p$, and total miners $=n$.
Number of ways to choose $k$ miners out of n is $n\choose k$.
Since the experiment for each node would be a Bernouli Trial, and success probability is same for each of them. \\

$$P(n,k) = {n\choose k}p^k(1-p)^{n-k}$$

\subsubsection*{(b)}
The probability that $n$ miners make attempt and are able to successfully mine $k$ blocks
(in fixed amount of time let's say T), with probability of success begin $p$, can be given by
$$P(k) = {n\choose k}p^k(1-p)^{n-k}$$
Let $r$ be the rate at which blocks are mined, so in $T$ time $T*r = \alpha \:...(i)$  blocks would be mined.
But by linearity of expectation we can say that number of blocks mined with $n$ attempts
(with success probability being $p$) is given by $p*n = \alpha \: ...(ii)$. \\

Using equation $(i), (ii)$ we get that $p*n = T*r$ \\
As $n \to \infty, p\to 0$ \\
or
\begin{align*}
    lim_{p\to \infty} P(k) &= lim_{p\to \infty} {n\choose k}p^k(1-p)^{n-k} \\
    & \text{using stirling apprximation and using $n = \frac{\alpha}{p}$ we get} \\
    lim_{p\to \infty} P(k) &= \dfrac{\alpha^ke^{-\alpha}}{k!}
\end{align*}
So probability that atleast one block would be mined in $T$ would be: $1-P(0) = 1 - e^\alpha = 1-e^{np} = 1 - e^{rT}$\\

Which means that time to mine next block follows exponential distribution. \\
Parameters of the distribution is the expected time of finding one block or $np$

\subsubsection*{(c)}
Taking timestamp as a paramter would not work since there is network latency, while broadcasting a block to
the community. Due to different timezones it would be difficult to have a synchronous time, ie we cannot have
the concept of global time. \\

Hence time stamp is not a good paramter to decide which block to add into the blockchain.

\subsubsection*{(d)}
We need to find value of $x$ that Bob should choose so that he can have $99\%$ confidence that $6$ blocks will be
found in the next $x$. Or in other words the probability to find atleast $6$ blocks in time $x$ should be $0.99$. \\

So $1 - \sum_{k=0}^{5}P(k) = 0.99$ \\
$$1 - \sum_{k=0}^{5} \dfrac{\alpha^k e^{-\alpha}}{k!} = 0.99$$
Using numerical analysis value of $\alpha \approx 13.1 = np = s*x$ \\
Assming block chain rate, we know $s = 0.1 block/min$ (10 min for 1 block), we get $x = 131$

\subsection*{Problem 6:}
Assuming that other miners have detected misbehaving miner's block, the next randomly selected node can
``boycott'', by not building on top of the misbehaving miner's block. In other words they can boycott
a particular address correspoding to the coinbase transaction of this block. \\

Since there is no real identity in
bitcoin blockchain, we cannot really identify the misbehaving node, based on this public key/wallet address.
Since detecting misbehaving node is hard, the misbehaving node can simply change his public key, and
can again misbehave. He wouldn't be affected much in this case, so ``boycott'' might not prevent him from misbehaving.

\subsection*{Problem 7:}
As a blockchain designer, we can change the puzzle from ``find a block whose hash is below certain value'' to
``find a block for which the hash of a signature on the block is below certain target''. So in this case
pool manager would have to share his private key with all the pool members, and this would be risky
since members might steal money from his wallet. Other alternative would be that pool manager does the signing work
and the members compute hash values. But since signing computationally more expensive than computing hash value,
this scheme would not work either. And we can prevent pool mining.

\subsection*{Problem 8:}
\begin{lstlisting}[style=C++]
pragma solidity ^0.5.0;

// Our contract name
contract MyFirstContract {
  // denotes account balance as uint
  uint public balance;
  
  // set balance to be 100 (initial value)     
  constructor() public {
    balance = 100;
  }

  // returns the current value of balance paramter of our contract
  function getBalance() public returns(uint) {
    return balance;
  }
}
\end{lstlisting}
\texttt{MyFirstContract.sol} \\


\begin{lstlisting}[style=C++]
// method to request a usable contract abstraction for a specific Solidity contract
var MyFirstContract = artifacts.require("./MyFirstContract.sol");

// Include this in exports.
module.exports = function(deployer) {
  // deploy this contract on Ethereum Network
  deployer.deploy(MyFirstContract);
};
\end{lstlisting}
\texttt{2\_deploy\_contracts.js} \\


\begin{lstlisting}[style=C++]
// Export the development configs.
module.exports = {
    networks: {
        development: {
        host: "localhost",     // local ethureum network
        port: 8545,            // Port of operation for ganache
        network_id: "*"        // * to match any network ID, it is a required field
        }
    }
}
\end{lstlisting}
\texttt{truffle-config.js} \\

\subsection*{Problem 9:}

Pros of Ethereum over Bitcoin are:
\begin{enumerate}
    \item Ethereum provide us with EVM, which allow code to be verified and executed on the blockchain. This would
          provide us guarantee that it will be run the same way on everyone's machine.
    \item There is no limit on block size, hence miners don't have to wait for block to fill, or
          remove some transaction in order to make block in size limit.
    \item Ethereum Platform provides people to run local instances for personal use, blockchain does
          not have such a feature.
    \item Block interval time in Ethereum is far less($10$ sec) than compared to that of bitcoin. On an
          average there are about $25$ transactions per second. Creating it more active form of currency.
    \item Ethereum provides us smart contract directly, without any requirement of tweaking in the case of
          Bitcoin.
\end{enumerate}

Cons of Ethereum over Bitcoin:
\begin{enumerate}
    \item Number of bitcoin is fixed to $21$ million, while number of Ethereum is around $91$ million,
          Having large number of coin in some sense means that value would be low. So Bitcoin can be a better option
          in terms of value.
    \item Bitcoin scripts are a bit restrictive(CFG based), so that is good for securing purpose.
    \item It is assumed that in Bitcoin, it is really different to find if 2 address are linked (belong to same owner etc).
          So this provides a higher degree of privacy. 
    \item There does not exist instances of bitcoin blockchain, as in the case of Ethereum. This would somehow
          prevent the value of Bitcoin blockchain from falling.
    \item Being first of its kind Bitcoin blockchain, has been the most popular among its peers. So it is more
          beneficial to use bitcoin.
\end{enumerate}

But we shouldn't really try to compare Ethereum and Bitcoin on the same scale,
as they are targeted for different applications.
\end{document}
