
%=======================   Default Templete   ==================
\documentclass[a4paper]{article}


% file with some default definations
\input{structure.tex}
\usepackage{listings}
\lstset{language=Python, basicstyle=\normalsize\sffamily\linespread{0.8}, numbers=left, numberstyle=\small, stepnumber=1, numbersep=5pt}
\usepackage{fancyhdr}
\usepackage{tikz}
\usepackage{tikz-qtree}
\usepackage{mathtools}
\DeclarePairedDelimiter{\ceil}{\lceil}{\rceil}

\setlength{\parindent}{0pt}

\pagestyle{fancy}
\fancyhf{}
\lhead{\textbf{\NAME\ (\ANDREWID)}}
\chead{\textbf{Mid Semester Exam \HWNUM}}
\rhead{\COURSE}

\renewcommand{\qedsymbol}{\rule{0.7em}{0.7em}}

%==================Header details======================
\newcommand\NAME{Raghukul Raman}
\newcommand\ANDREWID{160538}
\newcommand\HWNUM{}
\newcommand\COURSE{CS731}
%======================================================

% available formatted sections:
% - COMMAND LINE ENVIRONMENT: \begin{commandline} \end{commandline}
% - FILE CONTENTS ENVIRONMENT: \begin{file}[optional filename, defaults to "File"]
% - NUMBERED QUESTIONS ENVIRONMENT: \begin{question}[optional title]
% - WARNING TEXT ENVIRONMENT(can also be used for note): \begin{warn}[optional title, defaults to "Warning:"]
% - INFORMATION ENVIRONMENT(can be used to mention given details): \begin{info}[optional title, defaults to "Info:"]

%===============================================================
\begin{document}

\subsection*{Problem 1:}
Two suggested protocols to increase transaction rate to $70$/second are:
\begin{enumerate}
    \item Increase size of block from $1$MB to $10$MB.
    \item Decrease the block interval time to $1$ min.
\end{enumerate}
\textbf{Issues with increasing block size:} This would require a lot more storage, and would make the full
            nodes more expensive to operate. And eventaully fully operating nodes will decrease, and the system
            will become more centralized. There might be more double spending attacks due to slower propogation
            speeds. Achieving consensus would be difficult since validating a block would require lot more efforts. \\

\textbf{Issues with decreasing block interval time:} Propogation of the block in the network, takes some time. 
            If we reduce the block interval time, block might not be propogated fully in the network till that time.
            We might not be able to achieve consensus in due time. Block propogation and latency would also lead to more
            orphan nodes, since they might be delayed in propogating. It would also have an envionmental effect, since
            power would be used more since block rate is high.

\subsection*{Problem 2:}
See folder named \texttt{q2}

\subsection*{Problem 3:}
In bitcoin blockchain we can have double spending attacks. Let's see how: \\

Suppose Alice bought cocaine from Bob by paying him in bitcoin. Seeing this transaction in the most
recent block Bob might think that transaction is successful, and would give alway cocaine to Alice.
Now if the next random node this is selected in the next round happens to be controlled by Alice.
So she might ignore the block including her cocaine transaction, and could build on the second most recent block,
not including the cocaine transaction. So next time any miner would see 2 branches of equal length,
being honest it would the would be equally probable to build new block on either of them. Since There
is no way to distinguish that one of the block tries to double spend, there are approximately $50\%$ chances
that double spend would occur.\\
\textit{Note: Ideas of this answer are picked from section 2.3, bitcoin book}
% Doubt: should we explain 6 transaction confirmation or this would suffice?

\subsection*{Problem 4:}
\subsubsection*{(a)}
Since the miner is ahead of public blockchain by two secret blocks, all the mining efforts of the rest of the network
will be wasted. Other miners would mine on top of what they think is the longest chain, so after the
selfish miner announces, that branch would instantly become the new longest chain. Eventaully 
the rest of block (found by other miners), would become orphan. So in nutshell, gain in selfish mining
is that effective share of mining rewards would increase. \\
\textit{Note: Ideas of this answer are taken from section 5.5, bitcoin book.}

\subsection*{(b)}


\subsection*{Problem 5:}


\subsection*{Problem 6:}
Assuming that other miners have detected misbehaving miner's block, the next randomly selected node can
``boycott'', by not building on top of the misbehaving miner's block. Since there is no real identity in
bitcoin blockchain, we cannot really identify the misbehaving node, based on this public key/wallet address.
Since detecting misbehaving node is hard, the misbehaving node can simply change his public key, and
can again misbehave. He wouldn't be affected much in this case, so ``boycott'' might not prevent him from misbehaving.


\end{document}
